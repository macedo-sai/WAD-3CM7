

\documentclass[12pt]{article}


\usepackage{sbc-template}

\usepackage{graphicx,url}

\usepackage[brazil]{babel}   
%\usepackage[latin1]{inputenc}  
\usepackage[utf8]{inputenc}  
% UTF-8 encoding is recommended by ShareLaTex
\usepackage{verbatim}
\usepackage{listings}
\usepackage{xcolor}

\definecolor{verde}{rgb}{0,0.5,0}

%para customizar o código (ver https://en.wikibooks.org/wiki/LaTeX/Source_Code_Listings)
\lstset{language=C, %defina a linguagem usada no trabalho
              belowcaptionskip=1\baselineskip,
                breaklines=true,
                frame=false,
                xleftmargin=\parindent,
                showstringspaces=false,
                basicstyle=\footnotesize\ttfamily,
                keywordstyle=\bfseries\color{green!40!black},
                commentstyle=\itshape\color{purple!40!black},
                identifierstyle=\color{blue},
                stringstyle=\color{orange},
                numbers=left,
            }

\sloppy

\title{Estandar HTTP}

\author{Macedo Borbolla Eduardo Sai \inst{1}Grupo: 3CM5 }


\address{Escuela Superior de Cómputo  
\inst{1} Web Application Development
\email{\{mordrirsai\}@gmail.com}
}

\begin{document} 

\maketitle

\begin{resumo}
 Este escrito tiene como objetivo desribir de manera general la funcionalidad, el formato  y la utilidad de los identificadores de recursos uniformes así como sus localizadores
 y nombres dentro de una red de forma únivoca. 
\end{resumo}


\section{ URL, URI, URN: Cadenas de carácteres identificadoras}

Para la mejor gestión de contenido por medio de la web se crearon distintas maneras de específicar los nombres o identificadores de recursos que nos permite nombrarlos sin
probabilidad de poder confundir entre ellos o de dar una mal direccionamiento para accesar a alguno del cual ya sabemos su identidad, para poder resolver estas tres tareas
fueron creados tres diferentes cadenas de caractéres que nos permiten dar un nombre a los recursos según lo que necesitamos saber de ellos.

\begin{enumerate}
\item{URN: Nombre de recurso uniforme.  }
Nos permite definir de un espacio de nombres (un grupo de nombres o identificadores específicos) un nombre único para un recurso
en un contexto dado, es decir, dentro de un ambito cómo es posible llamar a un objeto sin ser confundido.7

Estructura:
\begin{itemize}
\item{URI-reference = URI / relative-ref}
\item{absolute-URI  = scheme ":" hier-part [ "?" query ]}
\end{itemize}

El mismo identificador puede ser declarado en múltiples espacios de nombres.
\item{URI: Identificador de recurso uniforme}
\begin{itemize}
\item{<scheme>:<scheme-specific-part>}
Donde el componente scheme en las comunicaciones TCP/Ip sobre la internet esta dado por protocolos de comunicación compatibles con HTTP.
\end{itemize}
Permite dar una identificación única a un recurso ya sea por su nombre o localización, en el caso de los URI nos pueden llevar a un URL para saber
el mecanismo principal de acceso al recurso nombrado.

\item{URL: Localizador de recurso uniforme}
Permite localizar a un recurso dado por un URI mediante su mecanismo principal de localización llamado query con el cual se solicita , a decir, nos lleva al recurso en cuestión pudiendo describir
su ubicación desde un contexto absoluto o relativo (entre otras opciones) al contexto donde se esta buscando.

Su estructura esta dada por la forma normal Backus-Naur descrita en el RFC 2141 "URN Syntax".


\end{enumerate}




\section{Conclusiones}
Los identificadores uniformes nos permite normalizar el acceso a la información, entidades y recursos pertinentes al sistema, pudiendo así usarlos
para poder definir valores permitidos dentro de nuestras aplicaciones web, como facilitar contenido a los usuarios, facilitando manejo de peticiones o incluso
prohibiendo el acceso a recursos no autorizados.
\end{document}
