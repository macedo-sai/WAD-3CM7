%%%%%%%%%%%%%%%%%%%%%%%%%%%%%%%%%%%%%%%%%%%%%%%%%%%%%%%%%%%%%%%%%%%%%%%%%%%
%
% Plantilla para un artículo en LaTeX en español.
%
%%%%%%%%%%%%%%%%%%%%%%%%%%%%%%%%%%%%%%%%%%%%%%%%%%%%%%%%%%%%%%%%%%%%%%%%%%%

\documentclass{article}

% Esto es para poder escribir acentos directamente:
\usepackage[latin1]{inputenc}
% Esto es para que el LaTeX sepa que el texto está en español:
\usepackage[spanish]{babel}

% Paquetes de la AMS:
\usepackage{amsmath, amsthm, amsfonts}
\usepackage[export]{adjustbox}
\usepackage{graphicx}
\graphicspath{{/home/euron/Documents/ESC/wad/practica1/images/}}

% Teoremas
%--------------------------------------------------------------------------
\newtheorem{thm}{Teorema}[section]
\newtheorem{cor}[thm]{Corolario}
\newtheorem{lem}[thm]{Lema}
\newtheorem{prop}[thm]{Proposición}
\theoremstyle{definition}
\newtheorem{defn}[thm]{Definición}
\theoremstyle{remark}
\newtheorem{rem}[thm]{Observación}

% Atajos.
% Se pueden definir comandos nuevos para acortar cosas que se usan
% frecuentemente. Como ejemplo, aquí se definen la R y la Z dobles que
% suelen representar a los conjuntos de números reales y enteros.
%--------------------------------------------------------------------------

\def\RR{\mathbb{R}}
\def\ZZ{\mathbb{Z}}

% De la misma forma se pueden definir comandos con argumentos. Por
% ejemplo, aquí definimos un comando para escribir el valor absoluto
% de algo más fácilmente.
%--------------------------------------------------------------------------
\newcommand{\abs}[1]{\left\vert#1\right\vert}

% Operadores.
% Los operadores nuevos deben definirse como tales para que aparezcan
% correctamente. Como ejemplo definimos en jacobiano:
%--------------------------------------------------------------------------
\DeclareMathOperator{\Jac}{Jac}

%--------------------------------------------------------------------------
\title{Práctica 1}
\author{Macedo Borbolla Eduardo Sai\\
  \small Configuración de entorno de dessarrollo web\\
  \small Web Application Development\\
}

\begin{document}
\maketitle

\abstract{Reporte de  la correcta instalacion y configuración de herramientas: Java, Tomcat y Apache para el curso de Web Application Development 19-1 }

\section{Entorno de desarrollo}

\subsection{Instalando Java}
Iniciamos el  modo  de usuario raíz para esta parte de la configuración
 \break Primero descargamos la versión de Java con la que desarrollaremos nuestro proyecto. \hfill
\break

\includegraphics[width=\textwidth]{p1st}
\newline
Siguiendo la configuración se extrae el archivo con tar -xvf 
\hfill
\break
\includegraphics[width=\textwidth]{p1tar}
\newpage
A continuación creamos una carpeta /usr/local/java (en este caso ya existente)

\includegraphics[width=	\linweidth]{p13}

\subsection{Subsection}\label{sec:nada}

Más texto.

\subsubsection{Subsubsection}\label{sec:nada2}

Más texto.

% Bibliografía.
%-----------------------------------------------------------------
\begin{thebibliography}{99}

\bibitem{Cd94} Autor, \emph{Título}, Revista/Editor, (año)

\end{thebibliography}

\end{document}